\documentclass[xcolor={table,dvipsnames}]{beamer}

\usetheme{Warsaw}

\usepackage[utf8]{inputenc}
\usepackage[frenchb]{babel}
\usepackage[T1]{fontenc}
\usepackage{amsmath}
\usepackage{hyperref}
\usepackage{multicol}

\usepackage{graphicx}

\usepackage{tikz}
\usetikzlibrary{arrows}

\pdfcompresslevel1

\newcommand{\camljava}{{\tt{camljava}}}

\usepackage{listings}
\usepackage{color}
\definecolor{darkgray}{rgb}{.5,.2,.2}
\definecolor{darkred}{rgb}{.8,.0,.0}

\usepackage[table]{xcolor}

\lstnewenvironment{OCaml}
                  {\lstset{
                      language=[Objective]Caml,
                      breaklines=true,
                      showstringspaces=false,
                      commentstyle=\color{red},
                      stringstyle=\color{darkgray},
                      identifierstyle=\ttfamily,
                      basicstyle=\small,
                      keywordstyle=\color{blue},
                      escapeinside={/*}{*/},
                      %xleftmargin=0.08\textwidth
                    }
                  }
                  {}


\lstnewenvironment{OCamlEx}
                  {\lstset{
                      language=[Objective]Caml,
                      breaklines=true,
                      showstringspaces=false,
                      commentstyle=\color{darkred},
                      stringstyle=\color{darkgray},
                      identifierstyle=\ttfamily,
                      keywordstyle=\color{blue},
                      basicstyle=\scriptsize,
                      escapeinside={/*}{*/},
                      frame=single,
                      %xleftmargin=0.08\textwidth
                    }
                  }
                  {}


\addtobeamertemplate{footline}{\hfill\insertframenumber/\inserttotalframenumber

\hspace{10em}\\}

\usepackage{listings}

\title{Typi : Application web de typage d'un mini-ML fonctionnel et impératif}
\author{Béatrice Carré \& Ghiles Ziat}
\institute{Projet de Typage et Analyse Statique}
\date{\today}
\titlegraphic{

   \includegraphics[scale=.4]{typi_logo.jpg}
   }

% slides number
\defbeamertemplate*{footline}{shadow theme}
{%
  \leavevmode%
  \hbox{
    \begin{beamercolorbox}[wd=.5\paperwidth,ht=2.5ex,dp=1.125ex,leftskip=.3cm plus1fil,rightskip=.3cm]{author in head/foot}%
    \usebeamerfont{author in head/foot}\insertframenumber\,/\,\inserttotalframenumber\hfill\insertshortauthor
  \end{beamercolorbox}%
  \begin{beamercolorbox}[wd=.5\paperwidth,ht=2.5ex,dp=1.125ex,leftskip=.3cm,rightskip=.3cm plus1fil]{}%
    \usebeamerfont{title in head/foot}\insertshorttitle%
  \end{beamercolorbox}}%
  \vskip0pt%
}

\beamertemplatenavigationsymbolsempty


\begin{document}

\maketitle







\begin{frame}{Objectif}


\begin{block}{Intérêt du typage}
associer à chaque expression un type dans le but de détecter des incohérences
\end{block}

\bigskip

\textbf{Les caractéristiques du typage} : 

  \begin{itemize}
  \item Typage statique ou dynamique
  \item Typage explicite ou implicite
  \item Typage fort ou faible
  \item L'inférence de type
  \end{itemize}


\end{frame}











\begin{frame}{La famille des langage ML}
Les langages dits ML sont des langages fonctionnels qui (du point de vue du typage) disposent classiquement des fonctionnalités suivantes:

\begin {itemize}
\item statiquement typés
\item fortement typés
\item inférence des types 
\end {itemize}
\medskip
On peut citer Caml, SML etc ...
\end{frame}




\begin{frame}{Le but du projet}
Application de typage d'un mini-ML en Ocaml.
\begin{itemize}
\item Lexing-Parsing construction d'un AST 
\item Typage du programme
\end{itemize}
\bigskip

Ajout de traits impératifs (références).
\begin{itemize}
\item Typage des références
\item Typage des références polymorphes
\end{itemize}
\end{frame}






\begin{frame}[fragile]{La grammaire de notre mini-ML}
 \begin{multicols}{2}
\begin{OCaml}
Expr: a ::= x
 | c
 | unop
 | binop
 | fun x -> a 
 | a/*$_1$*/ a/*$_2$*/ 
 | (a/*$_1$*/, a/*$_2$*/)
 | fst a
 | snd a
 | if a/*$_1$*/ then a/*$_2$*/ else a/*$_3$*/
 | let [rec] x = a/*$_1$*/ in a/*$_2$*/
\end{OCaml}
\noindent\small
identificateur (nom de variable)\\
constante\\
opération primitive unaire\\
opération primitive binaire\\
abstration de fonction\\
application de fonction\\
construction d'une paire\\
projection gauche\\
projection droite\\
alternative\\
liaison locale
\end{multicols}
\end{frame}













\begin{frame}[fragile]{Le système de type}

avec le système de type suivant :
\begin{multicols}{2}
\begin{OCaml}
Type : /*$\tau$*/ ::= /*$\cal{T}$*/

            | /*$\alpha$*/
            | /*$\alpha$*/ list
            | /*$\tau_1 \rightarrow \tau_2$*/
            | /*$\tau_1 \times \tau_2$*/

\end{OCaml}

\noindent
\small type de base ($int$, $float$, $bool$, $string$, $unit$)\\
variable de type\\
type de liste\\
type des fonctions de $\tau_1$ et $\tau_2$\\
type des paires de $\tau_1$ et $\tau_2$\\

\end{multicols}


\end{frame}










\begin{frame}[fragile]{Ajout de traits impératifs}
\begin{block}{Grammaire et système de type}
\begin{OCaml}
Expr : a ::= ... | l      /* adresse mémoire */ 
\end{OCaml}
\begin{OCaml}
Type : /*$\tau$*/ ::= ... | /*$\tau$*/ ref
\end{OCaml}
\end{block}
\pause

Opérateurs :
\begin{align*}
  ref \quad &: \quad \forall \alpha. \; \alpha \times \rightarrow \; ref  &\textrm{allocation}\\
  ! \quad &: \quad  \forall \alpha. \; \alpha \; ref \rightarrow \alpha  &\textrm{déréférencement}\\
  := \quad &:  \quad  \forall \alpha. \; \alpha \; ref \times \alpha \rightarrow unit  &\textrm{affectation}\\
\end{align*}
\end{frame}





\begin{frame}[fragile]{Les références polymorphes}
Considérons l'exemple :
\begin{center}
\begin{OCaml}
let r = ref (fun x -> x) in
   r := (fun x -> x+1);
  (!r) true
\end{OCaml}
\end{center}

\begin{itemize}
\pause
\item r reçoit $\forall \alpha. (\alpha \rightarrow \alpha)$
\pause
\item l'affectation sur r est bien typée
\pause
\item bloque sur $true+1$
\end{itemize}
\end{frame}



\begin{frame}

\begin{block}{Une solution}
S’assurer que l’on ne créera pas de référence dans une application. 
\end{block}

\bigskip

\begin{itemize}
\pause
\item \textbf{Expression non-expansive} : variables, constructeurs, abstraction... 
\pause
\bigskip
\item \textbf{Expression expansive} : application de fonction et application de $ref$
\end{itemize}
\end{frame}






\begin{frame}[fragile]{Les types faibles}

\begin{block}{Type faible}
Si l'expression est expansive, on ne généralise pas les types.
\end{block} 
\begin{itemize}
\item Indiquent un type particulier encore inconnu
\item pourront être instanciées ultérieurement.
\end{itemize}


\begin{OCamlEx}
type vartype = 
  Unknown of int
| Weak of int
| Instanciated of ml_type
\end{OCamlEx}

\end{frame}

\begin{frame}[fragile]{Exemple de type faible}
\begin{OCamlEx}
let f = ref (fun a -> a);;
/*val f : ('\_a -> '\_a) ref = \{contents = <fun>\}*/
\end{OCamlEx}
\pause
\begin{OCamlEx}
f := (fun x -> x+1);;
/*- : unit = ()*/
\end{OCamlEx}
\pause
\begin{OCamlEx}
f;;
/*- : (int -> int) ref = \{contents = <fun>\}*/
\end{OCamlEx}
\pause
\begin{OCamlEx}
!f true;;
/*Error: This expression has type bool but an expression was expected of type int*/
\end{OCamlEx}
\end{frame}

\begin{frame}{Typi}
TODO example impr
\end{frame}


\begin{frame}{Conclusion}

Typi :
\begin{itemize}
\item 
\end{itemize}

Travaux futurs:
\begin{itemize}
\item Gestion du typage explicite

\item Ajout de nouveaux traits (pattern matching, types somme ...)
\end{itemize}

\end{frame}












\begin{frame}{Bibliographie}

  \begin{thebibliography}{}
  \bibitem{DAOC} CHAILLOUX E., MANOURY P., PAGANO B., \emph{Développement
    d'applications avec Objective Caml}, O'Reilly
    , 2000


\bibitem{dea_leroy} LEROY X., \emph{} (\url{http://pauillac.inria.fr/~xleroy/dea/typage/cours.pdf})

\bibitem{remy} REMY D. (\url{http://gallium.inria.fr/~remy/isia/3/main.html})

\bibitem{jsofocaml} js\_of\_ocaml, OCSIGEN (\url{http://ocsigen.com/js_of_ocaml})

\bibitem{camlp4} CamlP4 \url{http://pauillac.inria.fr/camlp4/}
  \end{thebibliography}
\end{frame}







\end{document}

