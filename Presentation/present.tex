\documentclass[xcolor={table,dvipsnames}]{beamer}

\usetheme{Warsaw}

\usepackage[utf8]{inputenc}
\usepackage[frenchb]{babel}
\usepackage[T1]{fontenc}
\usepackage{amsmath}
\usepackage{hyperref}
\usepackage{multicol}

\usepackage{graphicx}

\usepackage{tikz}
\usetikzlibrary{arrows}

\pdfcompresslevel1

\newcommand{\camljava}{{\tt{camljava}}}

\usepackage{listings}
\usepackage{color}
\definecolor{darkgray}{rgb}{.5,.2,.2}
\definecolor{darkred}{rgb}{.8,.0,.0}

\usepackage[table]{xcolor}

\lstnewenvironment{OCaml}
                  {\lstset{
                      language=[Objective]Caml,
                      breaklines=true,
                      showstringspaces=false,
                      commentstyle=\color{red},
                      stringstyle=\color{darkgray},
                      identifierstyle=\ttfamily,
                      basicstyle=\small,
                      keywordstyle=\color{blue},
                      escapeinside={/*}{*/},
                      %xleftmargin=0.08\textwidth
                    }
                  }
                  {}


\lstnewenvironment{OCamlEx}
                  {\lstset{
                      language=[Objective]Caml,
                      breaklines=true,
                      showstringspaces=false,
                      commentstyle=\color{darkred},
                      stringstyle=\color{darkgray},
                      identifierstyle=\ttfamily,
                      keywordstyle=\color{blue},
                      basicstyle=\tiny,
                      escapeinside={/*}{*/},
                      frame=single,
                      %xleftmargin=0.08\textwidth
                    }
                  }
                  {}


\addtobeamertemplate{footline}{\hfill\insertframenumber/\inserttotalframenumber

\hspace{10em}\\}

\usepackage{listings}

\title{Typi : Application web de typage d'un mini-ML fonctionnel et impératif}
\author{Béatrice Carré \& Ghiles Ziat}
\institute{Projet de Typage et Analyse Statique}
\date{\today}
\titlegraphic{

   \includegraphics[scale=.4]{typi_logo.jpg}
   }

% slides number
\defbeamertemplate*{footline}{shadow theme}
{%
  \leavevmode%
  \hbox{
    \begin{beamercolorbox}[wd=.5\paperwidth,ht=2.5ex,dp=1.125ex,leftskip=.3cm plus1fil,rightskip=.3cm]{author in head/foot}%
    \usebeamerfont{author in head/foot}\insertframenumber\,/\,\inserttotalframenumber\hfill\insertshortauthor
  \end{beamercolorbox}%
  \begin{beamercolorbox}[wd=.5\paperwidth,ht=2.5ex,dp=1.125ex,leftskip=.3cm,rightskip=.3cm plus1fil]{}%
    \usebeamerfont{title in head/foot}\insertshorttitle%
  \end{beamercolorbox}}%
  \vskip0pt%
}

\beamertemplatenavigationsymbolsempty


\begin{document}

\maketitle







\begin{frame}{Objectif}


\begin{block}{Intérêt du typage}
associer à chaque expression un type dans le but de détecter des incohérences
\end{block}

\bigskip

\textbf{Les caractéristiques du typage} : 

  \begin{itemize}
  \item Typage statique ou dynamique
  \item Typage explicite ou implicite
  \item Typage fort ou faible
  \item L'inférence de type
  \end{itemize}


\end{frame}











\begin{frame}{La famille des langage ML}
Les langages dits ML sont des langages fonctionnels qui (du point de vue du typage) disposent classiquement des fonctionnalités suivantes:

\begin {itemize}
\item statiquement typés
\item fortement typés
\item inférence type 
\end {itemize}
\medskip
On peut citer Caml, SML etc ...
\end{frame}




\begin{frame}{Le but du projet}
Application de typage d'un mini-ML en Ocaml.
\begin{itemize}
\item Lexing-Parsing construction d'un AST 
\item Typage du programme
\end{itemize}
\bigskip

Ajout de traits impératifs (références).
\begin{itemize}
\item Typage des références
\item Typage des références polymorphes
\end{itemize}
\end{frame}






\begin{frame}[fragile]{La grammaire de notre mini-ML}
 \begin{multicols}{2}
\begin{OCaml}
Expr: a ::= x
 | c
 | unop
 | binop
 | fun x -> a 
 | a/*$_1$*/ a/*$_2$*/ 
 | (a/*$_1$*/, a/*$_2$*/)
 | fst a
 | snd a
 | if a/*$_1$*/ then a/*$_2$*/ else a/*$_3$*/
 | let [rec] x = a/*$_1$*/ in a/*$_2$*/
\end{OCaml}
\noindent\small
identificateur (nom de variable)\\
constante\\
opération primitive unaire\\
opération primitive binaire\\
abstration de fonction\\
application de fonction\\
construction d'une paire\\
projection gauche\\
projection droite\\
alternative\\
liaison locale
\end{multicols}
\end{frame}













\begin{frame}{}

\end{frame}











\begin{frame}{Application}
TODO example
\end{frame}


\begin{frame}{Conclusion}

O'Jacaré 2 :
\begin{itemize}
\item Accès simple à du code Java ou à l'API
\item Accès utilisateur transparent grâce aux classes encapsulantes
\item Gestion des tableaux simplifiée
\item Plus de possibilités grâce au sous-classage
\end{itemize}

Travaux futurs:
\begin{itemize}
\item Faire un paquet OPAM
\item Terminer la doc/manuel utilisateur
\end{itemize}
\bigskip
Merci à l'IRILL, G. Henry, X. Clerc, E. Chailloux.
\end{frame}












\begin{frame}{Bibliographie}

  \begin{thebibliography}{}
  \bibitem{DAOC} CHAILLOUX E., MANOURY P., PAGANO B., \emph{Développement
    d'applications avec Objective Caml}, O'Reilly
    , 2000

  \bibitem{}  HENRY G., \emph{O’Jacaré} \url{http://www.pps.univ-paris-diderot.fr/~henry/ojacare/}

  \bibitem{}  CLERC X.,\emph{OCaml-Java 2.0} \url{http://ocamljava.x9c.fr/preview/}

  \bibitem{} CHAILLOUX E., HENRY G., \emph{O’Jacaré, une interface objet
    entre Objective Caml et Java}, 2004

  \bibitem{} CLERC X., \emph{OCaml-Java: Typing Java Accesses from OCaml
    Programs}, Trends in Functional Programming, Lecture Notes in
    Computer Science Volume 7829,
    2013

  \bibitem{camljava} LEROY X., \emph{The camljava project},

  \bibitem{module Java} CLERC X.,\emph{OCaml-java : module Java} \href{http://ocamljava.x9c.fr/preview/javalib/index.html}{lien}

  \end{thebibliography}
\end{frame}







\end{document}

